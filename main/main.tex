\documentclass[9pt]{ltjsarticle}
\DeclareSymbolFont{bbold}{U}{bbold}{m}{n}
\DeclareSymbolFontAlphabet{\mathbbold}{bbold}
\newcommand{\bbold}{\mathbbold}
\usepackage{xcolor}
\usepackage{amsmath,amsfonts,amssymb}
\usepackage{enumitem}
\usepackage{ashiato45}
%\usepackage{okumacro}
\def\MARU#1{\textcircled{\scriptsize #1}}
\usepackage{graphicx}
\usepackage{ulem}
\usepackage{framed}
\usepackage{algorithm}
\usepackage{algorithmic}
\usepackage{here}
%\usepackage[twoside]{geometry}
\usepackage{mytheorems}
\usepackage{tikz}
\usepackage{ascmac}
\usepackage{stmaryrd}
\usetikzlibrary{cd}
\title{Tamura - Topology}
\author{ashiato45 take notes}
% \date{June 15, 2016}

\renewcommand{\bf}{\mathbf}


\begin{document}
\maketitle
\setcounter{section}{4}
\section{Application and example of homology}
\subsection{$\ker \psi = \Z$について}
$C_q(S^m\times \set{\pm 1})$について議論するために、これを構成する単体に
名前をつける。($I$との直積を取って柱を作ったときにやったと思います。)
$S^m$の単体$x$に対応する$S^m\times \set{\pm 1}$の単体を
$(x,\pm 1)$とよぶことにする。
一応の性質として、
\begin{itemize}
 \item 任意の$S^m\times \set{\pm 1}$の単体$x'$について、
       $S^m$の単体$x\in S^m$と$i\in \set{\pm 1}$が存在して、$y=(x,i)$と
       なる。
 \item 任意の$S^m$の単体$x,y$について、$(x,+1)\neq (y,-1)$。
\end{itemize}
がある。

また、$a=\sum_{x\colon S^mの単体}c_q \cdot x \in C_q(S^m\times \set{+1})$
について、
\begin{align}
 (a,+1) \defeq \sum_{x}c_q \cdot(x,+1)
\end{align}
と書くことにする。(単体の自由加群をそのまま$S^m\times \set{+1}$に持って
いった。)

$S^m\times \set{\pm 1}$のチェインが
\begin{align}
 \dots
 \xrightarrow{\pd'_{q+1}}
 C_q(S^m\times \set{\pm 1})
 \xrightarrow{\pd'_{q}}
 \dots
\end{align}
だとする。(いままでのは$\pd_*$で今回は$\pd'_*$。)

$\ker \pd'_q$のことを調べる。$a\in C_q(S^m\times), b\in
C_q(S^m)$とする。$C(S^m\times \set{\pm 1})$の任意の元は
$(a,+1)+(b,-1)$で代表できるので、これに$\pd'_q$をあてて0になったときを調
べる。
\begin{align}
 &
 \pd'_q((a,+1)+(b,-1)) = 0\\
 \desciff{S^m\times \set{\pm1}のつくりかた}&
 (\pd_q a, +1) + (\pd_q b, -1) = 0\\
 \desciff{$(x,+1)\neq (y,-1)$}&
 (\pd_q a, +1) = (\pd_q b, -1) = 0.
\end{align}
とする。よって、
\begin{align}
 \ker \pd'_q &= \set{(a,+1)+(b,-1) \in C_q(S^m\times \set{\pm 1}); 
 a,b\in \ker \pd_q}\\
 & =
\set{(a,+1); a\in \ker \pd_q} \oplus
 \set{(b,-1); b\in \ker \pd_q}.
\end{align}
よって、
\begin{align}
 H_q(S^m\times \set{\pm 1})
 &\desceq{定義}
 \ker \pd'_q / \Image \pd'_q\\
 &\desceq{さっきの}
\set{(a,+1); a\in \ker \pd_q}/ \Image \pd'_q \oplus
 \set{(b,-1); b\in \ker \pd_q}/ \Image \pd'_q\\
 &\simeq
 H_q(S^m)\oplus H_q(S^m)\\
 & \simeq
 \Z^2.
\end{align}
となっている。また、この代表元は$S^m\times \set{+1}$側からの
$[(a,+1)]$と$S^m\times \set{-1}$側からの$[(b,-1)]$のペア
$([(a,+1)],[(b,-1)])$とあらわせる。

これで$\psi_q$を調べる記号が整備できたのでやる。
$i\colon S^m\times \set{+1} \to S^m\times I$は複体の準同型で、
埋め込み(つまり、$i((a,+1))=(a,+1)$と定義されている。)
$j\colon S^m\times \set{-1}\to S^m\times I$も同様。

ここから、ホモロジー群の間の準同型
\begin{align}
 i_q \colon H_q(S^m\times \set{\pm 1}) \to H_q(H^m\times D_+^1)\simeq H_q(S^m\times I),\quad
 [(a,+1)] + [(b,-1)] \mapsto [(a,+1)]  + [(b,-1)]
\end{align}
と
\begin{align}
 j_q \colon H_q(S^m\times \set{\pm1}) \to H_q(H^m\times D_-^1)\simeq H_q(S^m\times I),\quad
 [(a,+1)] + [(b,-1)] \mapsto [(a,+1)]  + [(b,-1)]
\end{align}
が誘導される。このもとで、$\psi_q$は$\psi_q=(i_q,-j_q)$と定義されていた。

$\psi_q$の核を調べる。$a,b\in C_q(S^m)$とする。
\begin{align}
 &(i_q,-j_q)([(a,+1)] + [(b,-1)]) = 0\\
\iff& ([(a,+1)] + [(b,-1)], -[(a,+1)] - [(b,-1)]) = 0\\
 \iff &
 [(a,+1)] +[(b,-1)] = 0.
\end{align}
よって、
\begin{align}
 \ker \psi_q
 &=
\set{([(a,+1)],[(b,-1)])\in H_q(S^m\times \set{\pm 1});
 a,b\in C_q(S^m), [(a,+1)]+[(b,-1)] = 0}\\
 &=
\set{([(a,+1)],[-(a,+1)]); a\in C_q(S^m)}\\
 &\simeq 
\Z.
\end{align}



\end{document}
